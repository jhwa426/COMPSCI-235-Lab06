\documentclass{beamer}

\usepackage[utf8]{inputenc}
\usepackage{listings}
\usepackage{ulem}
\usepackage{hyperref}
\usepackage{amsmath}

\hypersetup{
	colorlinks=true,
	linkcolor=white,
	filecolor=magenta,
	urlcolor=cyan,
}

\usetheme{Dresden}

\title[COMPSCI235 Lab 4 (Week 6)]
{Flask configuration and blueprints, Jinja templates, WTForms}

\author{COMPSCI 235 Tutorial}


\date[July 2021]
{COMPSCI235 Lab 4 (Week 6)}

\begin{document}

	\frame{\titlepage}
	\section{Preparation}

	\begin{frame}
	  \frametitle{Lab Setup and Useful Links}
		\begin{itemize}
			\item \href{https://canvas.auckland.ac.nz/courses/60516/pages/lab-4-week-6-config-templates-blueprints-wtforms}{Canvas page with lab materials}
			\item This week lab journals will be marked. \href{https://canvas.auckland.ac.nz/courses/60516/assignments/246232}{Lab Journal Submission}
			\item The starter code for this week's lab is in \href{https://github.com/133caesium/COMPSCI-235-Lab-4-People/tree/lab4-task01-config}{this github repo}
			\item The solutions for each task are in a separate branch in the repository. You will learn more effectively if you attempt to solve each task yourself before looking at the solution.
			\item \href{https://github.com/martinurschler/2021CompSci235-03-CovidWebApp}{COVID-19 Web App Demo Project}
			\item \href{https://flask.palletsprojects.com/en/2.0.x/}{Flask Documentation} and \href{https://jinja.palletsprojects.com/en/3.0.x/}{Jinja Documentation}
			\item \href{https://wtforms.readthedocs.io/en/2.3.x/}{WTForms Documentation}
		\end{itemize}
	\end{frame}

	\begin{frame}
		\frametitle{Learning Intentions and Preparation}
		\begin{block}{Learning Intentions}
		Build, configure and run a simple Flask app that uses Jinja templating, a WTForm, and a Flask Blueprint.
		\end{block}
		\begin{block}{Related Content}
		\begin{itemize}
		\item Review the lecture materials for lectures 8 - 12.
		\item Familiarise yourself with the inital code. Draw some diagrams of the HTML page structures according to the supplied templates. Sketch a diagram to include the supplied project files and their relationships. Taking a few moments to do this helps you to visualise what you’re working worth and to review what’s been provided.
		\end{itemize}
		\end{block}
	\end{frame}



	\section{Lab Tasks}
	\begin{frame}
		\frametitle{Task 1: create configuration files }
		\begin{itemize}
			\item In the root directory of the project create a .env file and set the four key Flask configuration variables. (FLASK\textunderscore{}APP, FLASK\textunderscore{}ENV, SECRET\textunderscore{}KEY, and TESTING) Hint: \href{https://flask.palletsprojects.com/en/2.0.x/config/#configuring-from-environment-variables}{Flask Environment Documentation}
			\item Create a file named config.py in the root directory of the project. Use dotenv to read the .env file and to set the Flask configuration variables in the Config class. Hint:  \href{https://github.com/martinurschler/2021CompSci235-03-CovidWebApp}{See config.py in Demo Project}
			\item Assuming you have included FLASK\textunderscore{}ENV = ‘development’, the Web application is now configured to dump exception stack traces in the Web browser and to automatically reload the application in the Web server in response to any source code edits you make.
		\end{itemize}
	\end{frame}

	\begin{frame}
		\frametitle{Task 2: configure the Flask application object}
		\begin{itemize}
			\item In the people\textunderscore{}web\textunderscore{}app/\textunderscore{}\textunderscore{}init\textunderscore{}\textunderscore{}.py file, edit the create\textunderscore{}app() method to configure the newly created Flask object using the Flask configuration variables defined in config.Config Hint:  \href{https://github.com/martinurschler/2021CompSci235-03-CovidWebApp}{See \textunderscore{}\textunderscore{}init\textunderscore{}\textunderscore{}.py in Demo Project}
		\end{itemize}
	\end{frame}

	\begin{frame}
		\frametitle{Task 3: construct a simple homepage}
		\begin{itemize}
			\item The homepage should be named home.html and stored in the templates directory, and like the supplied list\textunderscore{}people.html template, it should extend layout.html. The homepage should override the block named content with something as simple.
		\end{itemize}
	\end{frame}

	\begin{frame}
		\frametitle{Task 4: construct a template for displaying a single person }
		\begin{itemize}
			\item This template should be named list\textunderscore{}person.html, and similarly to home.html and list\textunderscore{}people.html should extend layout.html. Assume that one variable (named person) will be supplied to list\textunderscore{}person’s content block; this is to be a Person object but could be a None (none in Jinja) value. If the person variable refers to a Person object (i.e. it’s not None), the template should output the Person’s attribute values, otherwise it should display a message to say that there is no matching Person (you’ll use this template later in task 10).
		\end{itemize}
	\end{frame}

	\begin{frame}
		\frametitle{Task 5: register the blueprint}
		\begin{itemize}
			\item A blueprint (defined in people\textunderscore{}web\textunderscore{}app/people\textunderscore{}blueprint/people.py) has been started. It is sufficiently fleshed out to be registered with the Flask application object. Add the three lines of code that are needed in create\textunderscore{}app() (in people\textunderscore{}web\textunderscore{}app/\textunderscore{}\textunderscore{}init\textunderscore{}\textunderscore{}.py). Hint:  \href{https://github.com/martinurschler/2021CompSci235-03-CovidWebApp}{See \textunderscore{}\textunderscore{}init\textunderscore{}\textunderscore{}.py in Demo Project}
			\item Try running the Web application now. You should be able to request the homepage from a browser.

		\end{itemize}
	\end{frame}


	\begin{frame}
		\frametitle{Task 6: implement the view method list\textunderscore{}people() }
		\begin{itemize}
			\item The people\textunderscore{}blueprint contains an empty method named list\textunderscore{}people() that is expected to handle requests for the entry point /list. Complete the implementation of this method by making a render\textunderscore{}template() call and returning the generated HTML page.

			\item In calling render\textunderscore{}template(), you’ll need to specify that the Web page should be generated by the list\textunderscore{}people.html template. This template expects to iterate over a collection of Person objects to populate the resulting Web page. Note that the repository is iterable, so you can simply pass the repository as the value for the people parameter defined by the list\textunderscore{}people.html template. The variable repo\textunderscore{}instance, defined in the repository module, refers to the repository.
		\end{itemize}
	\end{frame}

	\begin{frame}
	    \begin{itemize}
			\item Note too that since list\textunderscore{}people.html extends layout.html, and layout.html defines two parameters, your call to render\textunderscore{}template() should include suitable values for the parameters find\textunderscore{}person\textunderscore{}url and list\textunderscore{}people\textunderscore{}url that list\textunderscore{}people.html inherits from layout.html.

			\item Try running the application and requesting the homepage again. Now you should be able to click on the ‘List all leaders’ link and be presented with a Web page that lists all the Person objects stored in the repository.

		\end{itemize}
	\end{frame}

	\begin{frame}
		\frametitle{Task 7: prepare to use WTForms}
		\begin{itemize}
			\item Make sure that you set the SECRET\textunderscore{}KEY variable correctly in tasks 1 and 2. (You can check against the solution in the github repo)

			\item Edit the .env file to include variable WTF\textunderscore{}CSRF\textunderscore{}SECRET\textunderscore{}KEY and set it to an appropriate value. Refer to slide 8 of lecture 12. You don’t need to define this variable in the config.Config class – it just needs to be loaded into memory.
		\end{itemize}
	\end{frame}

	\begin{frame}
		\frametitle{Task 8: make a WTForm class}
		\begin{itemize}
			\item Write a new subclass of WTForm named SearchForm.
			\item SearchForm should define two fields, an IntegerField, referred to by attribute person\textunderscore{}id, and whose name/label is ‘Person id’. This field is a mandatory field. The second field should be a SubmitField, referred to by attribute submit and whose name/label is ‘Find’.
			\item You should place this class declaration in the people\textunderscore{}blueprint\people.py file.
            \item Hint:  \href{https://github.com/martinurschler/2021CompSci235-03-CovidWebApp/blob/main/covid/news/news.py}{See the form in news.py in the Demo Project}

		\end{itemize}
	\end{frame}

	\begin{frame}
		\frametitle{Task 9: construct a template for a search form}
		\begin{itemize}
			\item This template should be named find\textunderscore{}person.html and should extend layout.html.
			\item The purpose of the form is to allow users to enter a person’s ID and then to post it to the Web app.
			\item The content block in find\textunderscore{}person.html should be overridden, and include the following variables in Jinja expressions:
			\begin{itemize}
			    \item handler\textunderscore{}url, the URL to post the form to.
                \item form.csrf\textunderscore{}token, a hidden field that stores the form’s csrf token.
                \item form.person\textunderscore{}id.label, the label associated with the person\textunderscore{}id field.
                \item form.person\textunderscore{}id, the integer field used to type a person’s ID.
                \item form.submit, the submit field
            \end{itemize}
		\end{itemize}
	\end{frame}

	\begin{frame}
		\frametitle{Task 10: complete the blueprint by implementing view method find\textunderscore{}person()}
		\begin{itemize}
			\item All that remains is to flesh out method find\textunderscore{}person() in the blueprint. Recall from lectures that all view methods that handle requests to retrieve (GET) and process (POST) forms follow the same pattern:
            \begin{itemize}
                \item Instantiate a WTForm subclass.
                \item Call validate\textunderscore{}on\textunderscore{}submit() on the form object. If this method returns True, it means that the HTTP request is a POST and any validation associated with the form has passed.
            \end{itemize}
    	\end{itemize}
	\end{frame}

	\begin{frame}
	    \begin{itemize}
            \item For this Web app, you should write the few lines of code to:
            \begin{itemize}
                \item Extract the POSTed ID value from the form by accessing the form’s person\textunderscore{}id.data attribute.
                \item Query the repository for a Person whose ID matches the value stored in person\textunderscore{}id.data.
                \item Make a call to render\textunderscore{}template to generate a webpage that either displays the Person (if found) or displays a message saying that there is no match. Use the template you developed in task 4, and provide values for the three named parameters: find\textunderscore{}person\textunderscore{}url, list\textunderscore{}people\textunderscore{}url, and person. Then, return the generated HTML page.
            \end{itemize}
            \item In cases where validate\textunderscore{}on\textunderscore{}submit() returns False, this means that either the HTTP request is POST and validation failed, or that the HTTP request is a GET for the bank form. In either case, the form needs to be returned as the HTTP response.
        \end{itemize}
	\end{frame}

	\begin{frame}
        \begin{itemize}
            \item For this Web app, you simply need to make a render\textunderscore{}template() call to generate the Web page containing the form using the find\textunderscore{}person.html template you developed in task 9. You should supply values for the four named parameters: find\textunderscore{}person\textunderscore{}url, list\textunderscore{}people\textunderscore{}url, handler\textunderscore{}url and form.

		\end{itemize}
	\end{frame}


\end{document}